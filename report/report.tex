%  PHDTHESIS TEMPLATE FILE
%  Adopted from Thomas Fabricius and Henrik Aalborg Nielsen
%  Jan Larsen, IMM, DTU, Nov 2003 ver 1.0
%  Updated by Finn Kuno Christensen, fkc@imm.dtu.dk Aug 15, 2008

%  COMPILATION STEPS USING INVOLVING A PS FILE
%\documentclass[10pt,twoside,dvips]{book}
%  latex phdthesis.tex
%  dvips -D600 -Pamz -Pcmz -j0 phdthesis.dvi -o phdthesis.ps
%  ps2pdf -sPAPERSIZE=b5 phdthesis.ps phdthesis.pdf (or use Acrobat Distiller)

%  COMPILATION STEP USING PDFLATEX
\documentclass[11pt,twoside]{book}
%  pdflatex phdthesis.tex

\usepackage[utf8]{inputenc}
\usepackage{url}
\usepackage{python}
\usepackage[round]{natbib}
\usepackage{color}
\usepackage{caption}
\captionsetup{font={small,it},labelfont=bf}
\usepackage{subfig}
\usepackage{paralist}
\definecolor{mygray}{RGB}{244,244,244}
\definecolor{gray}{gray}{0.5}
\definecolor{myredish}{RGB}{193,33,97}
\definecolor{grayblue}{RGB}{91,112,142}
\definecolor{myorange}{RGB}{255,134,0}
\definecolor{green}{rgb}{0,0.4,0}
\usepackage{tabularx}

\usepackage[pdftex]{hyperref}
\hypersetup{ 
    colorlinks,% 
    citecolor=grayblue,% 
    filecolor=grayblue,% 
    linkcolor=grayblue,% 
    urlcolor=grayblue 
}

\newcommand\mydef[1]{\emph{#1}}
\newcommand\mytodo[1]{\textbf{\textcolor{red}{TODO: #1}}}
\newcommand\fn[1]{{\ttfamily\small #1}}
\newcolumntype{R}{>{\raggedleft\arraybackslash}X}%
\newcommand\appref[1]{appendix \ref{app:#1}}
\newcommand\mypagebreak{\vspace{\stretch{1}}\pagebreak}
\newcommand\githuburl[1]{\url{https://github.com/alphabits/ford_challenge/tree/master/#1}}
\newcommand\githublisting[2]{\lstinputlisting[caption=\githuburl{#2}]{#1#2}}
\newcommand\norm[1]{||\,#1\,||}
\newcommand\ve[1]{\boldsymbol{#1}}
\newcommand\varop[1]{\text{Var}\,[#1]}
\newcommand\expop[1]{\text{E}\,[#1]}
\newcommand\covop[1]{\text{Cov}\,[#1]}
\newcommand\myimp{\quad\quad\Leftrightarrow}
\newcommand\myreal{\mathbb{R}}
\newcommand\R{\mathbb{R}}
\newcommand\TFC{The Ford Challenge}
\newcommand\Ptx{P(t|\ve{x})}
\newcommand\PtxHat{\widehat{P}(t|\ve{x})}
\newcommand\mle{\widehat{\ve{w}}}
\newcommand\Lx{\ve{w}^T\ve{x}+w_0}
\newcommand\Err{\text{Err}}
\newcommand\err{\overline{\text{err}}}
\newcommand\mytr[1]{\small\Tr{#1}}
\newcommand\CI[2]{\mathbb{P}(\mu_{AUC} \in [#1, #2]) = 95\%}

\renewcommand\thefootnote{\fnsymbol{footnote}}

\usepackage{lmodern}
\usepackage{fouriernc}
\usepackage{eucal}
\usepackage[scaled]{beramono}
\usepackage{listings}
\lstset {                 % A rudimentary config that shows off some features.
    language=Python,
    basicstyle=\scriptsize\ttfamily, % Without beramono, we'd get cmtt, the teletype font.
    commentstyle=\textit, % cmtt doesn't do italics. It might do slanted text though.
    keywordstyle=\bfseries,
    framextopmargin=4pt,
    framexbottommargin=4pt,
    framexleftmargin=4pt,
    framexrightmargin=4pt,
    xleftmargin=4pt,
    xrightmargin=4pt,
    backgroundcolor=\color{mygray},
    frame=single,
    showstringspaces=false,
    captionpos=b,
    tabsize=4            % Or whatever you use in your editor, I suppose.
}

\renewcommand{\lstlistlistingname}{Code Listings}
\renewcommand{\lstlistingname}{Code Listing}

\lstset{
    language=python,
    stringstyle=\color{myorange},
    showstringspaces=false,
    alsoletter={1234567890},
    otherkeywords={\ , \}, \{},
    keywordstyle=\color{blue},
    emph={access,and,break,class,continue,def,del,elif ,else,%
    except,exec,finally,for,from,global,if,import,in,i s,%
    lambda,not,or,pass,print,raise,return,try,while},
    emphstyle=\color{myredish}\bfseries,
    emph={[2]True, False, None, self, __file__},
    emphstyle=[2]\color{green},
    emph={[3]from, import, as},
    emphstyle=[3]\color{blue},
    upquote=true,
    morecomment=[s]{"""}{"""},
    commentstyle=\color{gray}\slshape,
    emph={[4]1, 2, 3, 4, 5, 6, 7, 8, 9, 0},
    emphstyle=[4]\color{blue},
    literate=*{:}{{\textcolor{blue}:}}{1}%
    {=}{{\textcolor{blue}=}}{1}%
    {-}{{\textcolor{blue}-}}{1}%
    {+}{{\textcolor{blue}+}}{1}%
    {*}{{\textcolor{blue}*}}{1}%
    {!}{{\textcolor{blue}!}}{1}%
    {(}{{\textcolor{blue}(}}{1}%
    {)}{{\textcolor{blue})}}{1}%
    {[}{{\textcolor{blue}[}}{1}%
    {]}{{\textcolor{blue}]}}{1}%
    {<}{{\textcolor{blue}<}}{1}%
    {>}{{\textcolor{blue}>}}{1},%
    rulesepcolor=\color{blue}
}



%% Define a new 'leo' style for the url package that will use a smaller font.
\makeatletter
\def\url@leostyle{%
  \@ifundefined{selectfont}{\def\UrlFont{\sf}}{\def\UrlFont{\small\ttfamily}}}
\makeatother
%% Now actually use the newly defined style.
\urlstyle{leo}


%%%%%%%%%%% MODIFY THESE LINES ONLY %%%%%%%%%%%%%%%%%%%%%%%%%%%%%%%%%%%%%%%%%%%%%%%%%%%%%%%%%
\def\thesisyear{2011} % Year thesis submitted
\def\thesisnumber{70}  % Only number no year
\def\thesisauthor{Anders Hørsted} % Thesis author
\def\thesistitle{Detection of human alertness using supervised learning} % Title of thesis
\def\thesiskeywords{mathematical modelling}
\def\thesisISBN{} %OBSOBS provide ISBN number for industrial phd students ONLY
\def\thesisversion{print} %OBSOBS choose this for printed version send to printing
%\def\thesisversion{net} %OBSOBS choose this for the net version for the web and publication database
%%%%%%%%%%%%%%%%%%%%%%%%%%%%%%%%%%%%%%%%%%%%%%%%%%%%%%%%%%%%%%%%%%%%%%%%%%%%%%%%%%%%%%%%%%%%%

%%%%%%%%%%%%%%% DO NOT MODIFY START %%%%%%%%%%%%%%%%%%%%%%%%%%%%%%%%%%%%%%%%%%
\input{immphddef.sty}
%%%%%%%%%%%%%%% DO NOT MODIFY END %%%%%%%%%%%%%%%%%%%%%%%%%%%%%%%%%%%%%%%%%%%%



%%%%%%%PREFACE CHAPTERS INCLUDE%%%%%%%%%%%%%%%%%%%%%%%%%%%%%%%%%%%%%%%%%%%%%%

\chapter*{Summary}

In this project, the logistic regression model used to win the online machine learning competition ``Stay Alert! The Ford Challenge'' is recreated, using the Python machine learning library scikits.learn. Performance similar to the performance of the winning model is achieved by the new model. \par
By doing a forward feature selection, a model is found that uses the same number of features but consistently performs better than the winning model. \par
Finally it is tried to further enhance the performance by training a neural network model in the Python library PyBrain. No significant improvements are achieved by the neural network, compared to the logistic regression, but this is in part attributed to lack of computational resources.

\markboth{}{}
\chapter*{Resum\'e} %{Sammenfatning (Summary in Danish)}

På dansk

\markboth{}{}
\chapter*{Preface}

This project is the result of attending the course ``01666 Project work - Bachelor of Mathematics and Technology'' at the Technical University of Denmark. The course is mandatory for all Bachelors at Mathematics and Technology and counts for 10 ECTS points. \par

The subject of the project is held within the discipline called machine learning that, depending on how it is defined, can be seen as part of artificial intelligence or applied statistics. \par

The target group of the report is any student that have completed the course ``01005 Mathematics 1'' at DTU or a similar course at another university.

All files used in this project are available at \githuburl{ }.

\vspace{20mm}
Anders Hørsted \\
Christianshavn, June 2011

\markboth{}{}
\chapter*{Acknowledgements}

I would like to thank my two supervisors Morten Mørup and Ole Winther, for the great help they have given me during this project.
At no time have I had trouble getting help when I had questions, even when I showed up unannounced. It has been a great pleasure to have Morten and Ole as supervisors.

\markboth{}{}

%%%%%%%PREFACE CHAPTERS INCLUDE%%%%%%%%%%%%%%%%%%%%%%%%%%%%%%%%%%%%%%%%%%%%%%


\newpage\mbox{}\newpage
\chaptermark{Contents}
\renewcommand{\sectionmark}[1]{\markright{#1}}
\sectionmark{Contents}
\addtolength{\parskip}{-\baselineskip}
\tableofcontents
\addtolength{\parskip}{\baselineskip}
\renewcommand{\sectionmark}[1]{\markright{\thesection\ #1}}

\mainmatter
% Chapter 1, 2, ...
%%%%%%%MAIN CHAPTERS INCLUDE%%%%%%%%%%%%%%%%%%%%%%%%%%%%%%%%%%%%%%%%%%%%%%

\part{Introduction and data exploration}
\chapter{Introduction}

This decade could very well be the most exciting decade for anyone interested in data analysis. The ease with which data can be collected, combined with the very cheap storage prices makes it likely that before the end of this decade, most details about a person's life will in some way be collected digitally. The increasing amount of data calls for a continous refinement and advancement of both the computational and statistical techniques used to analyze the data. As it is exactly this borderland between statistics and computer science, that is the focus area of machine learning, it seems probable that much of the advancement in the near future will happen within the machine learning community. This of cause, makes machine learning an interesting study area and it wasn't a difficult decision to choose machine learning as the general topic for this project. \par

Although the general topic was chosen, the question still remained about what should be the specific subject of the project. The idea soon emerged that the project could be based on one of the many online machine learning competitions, that have appeared within the last couple of years. The idea behind these online competitions, is that some private or public organization, releases a dataset along with a specific task to be solved using the dataset. One such competition, called The Ford Challenge, was announced in the beginning of January and as it was an exciting competion it was chosen as the subject for this project. \par

In The Ford Challenge a model must be build, that can predict when a driver is about to become inattentive. By using data collected from various drivers it should be possible to let a computer detect patterns that are related to an inattentive driver. The task is indeed exciting but as the competition already ended in the beginning of March, active participation in the competition couldn't be the main focus of this project. Instead it was chosen to try and recreate the model of the Ford Challenge winner. When a performance similar to that of the winner was achieved, various techniques to improve on the winning model could be tried. Before any model could be build though, some thorough data exploration had to be done, as almost no information about the dataset was revealed by Ford. \par

Apart from the goals specifically related to The Ford Competition, it was also strongly desired to set up a working environment for data analysis, based solely on open source tools. When these tools were found some time should be used to set up a flexible and simple working environment. \par

The problem statement of this project is:
\begin{itemize}
    \item Try to decide the datatype for all features in the dataset, by doing a thorough data exploration
    \item Try to recreate the winning model of The Ford Competition.
    \item Improve on the winning model by using other models than the winner.
    \item Set up a working environment for data analysis, and reflect on possible ways to ease future data analysis projects.
\end{itemize}

Much time have went into solving the above problem statement, but much time was also spent working on this report. The report starts out by presenting the Ford Challenge, along with descriptions of some other machine learning competitions. After introducing the Ford Challenge, the dataset of the Ford Challenge is introduced in details. Then the data is thoroughly explored and some time is spent trying to decide the datatype of the various features. After the data exploration, a chapter about the theory of binary classification follows. The theory is kept at a basic level and should only be used as a quick summary. After the theory comes the modelling chapters. Results obtained from the various experiments are presented and a basic confidence interval is calculated. All results are summarized in the last chapter of the modelling section. Following the modelling section is a chapter about project management and the software used for this project. Normally this chapter would have been an appendix, but since setting up a working environment is part of the problem statement a chapter is devoted to this subject. Finally the report closes with a discussion, conclusion and appendices.

\chapter{The Ford Challenge} 
In this chapter, the ford competition that is the basis for this report, is introduced. Before introducing the ford challenge, a short overview of other online machine learning competitions is given. As part of introducing the ford competition, the kaggle.com website that hosted the ford competition is presented, and the data set used in the ford competition is described in detail. But as metioned the chapter starts with a short overview of other online machine learning competitions.

\section{Other online machine learning competitions}
To get a little perspective on the ford challenge, this section gives a short review of some past and present online machine learning competitions.

\subsection{Netflix Prize}
One of the most talked about competitions, may very well be the Netflix Prize. The Netflix competition was launched on October 2, 2006 and the aim of the competition was to predict how users would grade new movies, based on a large dataset of previous grades. Why did the Netflix Prize gather a lot of attention? First of all the grand prize was \$1M, which is a lot of money. And secondly the dataset was huge, consisting of 100,480,507 ratings given by 480,189 users, leaving room for a lot of interesting new techniques to be tested. \par 
When the Netflix Prize was awarded on September 18, 2009, 5169 different teams had submitted at least one entry to the competition. The winning team consisted of three different teams, that at one point decided to team-up and compete as a join-team.\par 
Netflix originally wished to follow up the Netflix Prize, with another competition but decided to dismiss the idea, due to a lawsuit regarding privacy concerns related to the first Netflix Prize. \footnote{The section about The Netflix Prize is based on \citep{wiki:netflix_prize} and \citep{netflix_leaderboard}}

\subsection{KDD Cup}
Although the Netflix Prize gather a lot attention, it wasn't the first online machine learning competition. An example of an earlier competition, is the KDD Cup. The KDD Cup is a competition that is held every year, and that started back in 1997. The subject changes every year, and can be anything from mining purchase data from an online store, to computer aided detection of breast cancer (the 2000 and 2008 competitions respectively).\par
This year the KDD Cup is held in coorporation with Yahoo! Labs and the task is to predict user ratings of musical items (both tracks, albums, artists and genres). One note worthy detail about the 2011 KDD Cup is the huge data set, containing over 300 million ratings of more than 600,000 distinct items. \footnote{Read more about the KDD Cup history at \citep{kdd_cup_center}. For more info about the 2011 competition see \citep{kdd_cup_2011}}

\subsection{And many others}
The Netflix Prize and the KDD Cup are just two examples of online machine learning competitions. Many others exists, such as
\begin{itemize}
    \item \emph{The Hearst Challenge 2011} - Every year The Hearst Corporation hosts a machine learning competition. This year the task is to data mine the history of 1.8 million emails sent to subscribers of Hearst's publications, and then predict who will open emails in the future. \\
        Read more at \url{http://www.hearstchallenge.com}
    \item \emph{The Reclab Prize} - RichRelevance is a company that specializes in online product recommendation. They offer a \$1M prize for the team that first improves their product recommendation algorithm by 10\%. \\
        Read more at \url{http://www.overstockreclabprize.com}
    \item \emph{The Heritage Health Prize} - 
\end{itemize}
Since almost all the machine learning competition websites, needs the same functionality, websites that specializes in hosting machine learning competitions has appeared. One example of a website that provides a hosting platform for machine learning competitions is the kaggle.com website, who hosts The Ford Challenge.

\section{Kaggle.com and The Ford Challenge}
As already mentioned, kaggle.com hosts machine learning competitions for universities and corporations. The first competition hosted by kaggle.com was started in April 2010, and since then a total of 18 competitions have been held. Every competition at kaggle.com has some background information, links to the data sets, a submission system and a forum, as seen in figure \ref{fig:fordchallenge_frontpage}. \par

\begin{figure}[tbhHp]
    \centering
        \includegraphics[width=.9\textwidth]{media/fordchallenge_frontpage.png}
    \caption{The Ford Challenge frontpage at the kaggle.com website}
    \label{fig:fordchallenge_frontpage}
\end{figure}

The Ford Challenge began on January 19, 2011. The task was to create a classifier that is able to detect when a driver is about to get distracted while driving. The dataset that Ford made available for the competition consisted of measurements of 30 different features, measured on drivers along with a binary feature (IsAlert) that was 1 if the driver was alert and 0 otherwise. The 30 features was a mix of environmental, driver physiological and vehicular features. Based on this dataset a classifier should predict the IsAlert feature of a distinct test dataset held by Ford. \par
One detail that made the competition slightly different than many other competitions, was that Ford would not disclose any information about what the different features represented\footnote{see forum replies from the Ford spokesperson \citep{kaggle_forum_266,kaggle_forum_317}}. The official reason was that (see \citep{kaggle_forum_268_reply_2})
\begin{quote}
    ``We like to encourage the participants to pursue classification without preconceived notions based on prior knowledge of the subject, focusing on variables which lead them (based on their own experiments) to better classification."
\end{quote}
Doubts about the true motive behind the lack of details about the features, was expressed by, what later turned out to be, the winner of the competition (see \citep{kaggle_forum_295_reply_3}) \par

The performance of the classifiers was measured by calculating the AUC (see section \mytodo{AUC-section}) of the classifiers, on the test set. A limit of two submissions pr contestant was set as a way to counteract the possibility of someone reverse-engineering the IsAlert-feature of the test dataset.

\subsection{The data set}



\chapter{Data exploration}
Here I describe the various data exploration techniques I have used. Lots of nice graphs. PCA. Boxplots. Feature plots. Scatter Plots. Unique Values. Is it discrete, binary, continuous.

\section{A note about Test data, training data}

\section{Calculating common statistics}
To start the data exploration, four common statistics, namely the mean, min, max and standard deviation, of every feature across the whole dataset was calculated. The standard deviation was calculated as (with $n$ equal the total number of rows in the dataset, and $x_i$ equal to the $i$'th value of any of the features)
\[
    s = \sqrt{\frac{\sum_{i=1}^n (x_i-\bar{x})^2}{n}}
\]
The source code for the calculations can be found in \appref{source-common-statistics} and all results in \appref{result-common-statistics}. Most results did not tell that much, but a few results stood out. These are shown in table~\ref{tbl:summary-stats-highlights}.
\begin{table}
    {\small\sffamily
        \begin{python}
            import scripts.commonstats_table as c; c.render('../sessions/9-data-exploration/src/summary_statistics.json', ['V5', 'V7', 'V9', 'E9', 'P8', 'P6', 'IsAlert'])
        \end{python}
    }
    \caption{Highlights from the results of the summary statistics. See~\appref{result-common-statistics} for all results.}
    \label{tbl:summary-stats-highlights}
\end{table}
The table shows that the features \fn{P8}, \fn{V7} and \fn{V9}, are zero throughout the whole dataset, and can be ignored. The features \fn{E9} and \fn {V5}, could be binary, but that is only speculation at the moment. Finally the feature \fn{P6} has a mean of 843.73 and a standard deviation of 2795.32, but its maximum is 228812.00, which is many, many standard deviations away from the mean. This could be a sign of some serious outliers, but it could as well be a single trial with a mean far from the other trials. Further investigation are needed to conclude anything here. Finally it is seen that the mean of the \fn{IsAlert} feature is only a little above 0.5, and therefore only a little over 50\% of the time are the drivers alert. This may be a bit surprising. \par
The results of this first step have been to exclude a few features, and get some rough ideas about the shape of some other features. It is now time to resally get to know the different features.

\section{Determining the datatype of features}
Since Ford would not disclose any information about the different features, it is important to get a good picture of which features are discrete/categorical and which features are continous. A natural first step to learn the datatype of the features, is to calculate the number of unique values each feature takes.
\subsection{Unique values}
It requires a little thought to pinpoint exactly what needs to be calculated. On one hand it is natural to calculate the number of unique values a feature takes across the whole dataset. On the other hand it could happen, that a discrete feature might have only (eg.) 3 unique value within any trial, but the values differ between various trials. This way we would have a feature with 1500 unique values across the whole dataset, but with max 3 unique values within any given trial. Therefore the number of unique values within a trial are calculated for every feature and every trial. Based on this result, the minimum and maximum number of unique values within a single trial, is calculated for every feature. The source code can be found in \appref{source-unique-values} and results are shown in table~\ref{tbl:result-unique-values}.\mytodo{The result table in appendix or in report?} \par
\begin{table}
    {\small\sffamily
        \begin{python}
            import scripts.uniquevalues_table as c; c.render(['V7','V9','P8'])
        \end{python}
    }
    \caption{The minimum and maximum number of unique values within the trials, for every feature in the dataset. Also the total number of unique values for each feature, across the whole dataset, are shown.}
    \label{tbl:result-unique-values}
\end{table}
A couple of things should be noticed about the results. Almost all features have som trials where they only take on one unique value. For categorical features, this could be perfectly normal, but for a continous feature it seems to be pretty unlikely to have only one value in a trial spanning two minutes. Is a feature, that in some trials seems continous, but then only have one value in other trials, just turned off in the latter trials? And how should the feature be handled in trials where it is ``off"?. This discussion is continued in section~\ref{sec:outlier-detection} about outlier detection. \par

From table~\ref{tbl:result-unique-values} it is also seen, that only features \fn{P1}, \fn{P2} and \fn{V11} is consistently having lots of unique value, across all trials. It seems fair to call these features continous. Feature \fn{V1} have some trials where it takes on 969 different values, but it also have trials with only one unique value. Across the whole dataset it takes on 12374 unique values. This could be explained by feature \fn{V1} being continous in a small number of trials (about 15-30), and turned ``off" in all other trials. Further exploration will show whether it is true or not. \par

Another interesting detail is that there are two pairs of features ((\fn{P3},\fn{P4}), (\fn{P6},\fn{P7})) in the results, that share exactly the same number of unique features. Both min and max and total. This indicates a possible relationship between the features, and later (section~\ref{sec:scatterplots}) it is shown that this is in fact true. \par

A final remark about the number of unique values, is that feature \fn{E9} and \fn{V5} indeed are binary. Also if a limit of max 40 unique values within a single trial is set, as an indicator for categorical features, \fn{E3}, \fn{E7}, \fn{E8}, \fn{V3} and \fn{V10} are seen to be categorical. \par

\subsection{Plotting some features}
P6 in trial 372, 374 is very strange.


\section{Outlier detection}\label{sec:outlier-detection}
Broader view than 95\% percentile. Trial with only IsAlert=0 outlier? Maybe mention troubles with defining outlier? Theorem in book Duda.
\subsection{Making boxplots of features}

\section{Finding possible discriminating features}
\subsection{Testing binary features}
\subsection{Scatterplots}\label{sec:scatterplots}
\subsection{Making a Principal Component Analysis}
Giver det mening på binære variable?

\section{Conclusions}

\part{Modelling}
\chapter{Theory of classification}
In this chapter some of the theory of classification is explained. The discussion is constrained to binary classification, of which the Ford Challenge classifier, is an example. First a general problem definition is given and notation is introduced. Then three different approaches to classification are introduced, and some pros and cons of each approach are mentioned. After this two concrete examples of classification models are introduced (Logistic Regression and Feed Forward Neural Network), and they are related to the three general approaches of classification. Finally different ways to measure the performance of different classifiers are discussed, including AUC that was used as the grading method in The Ford Challenge.

\section[The binary classification problem]{The binary classification problem\protect\footnote{This section is loosely based on \citet[sec 22.1-22.2]{wasserman04}}}
In a binary classification problem a binary outcome $t$ must be predicted from a $d$-dimensional input vector $\ve{x}=[x_1, x_2, \dots, x_d]^T$. The input vector represents an event, and the outcome variable $t$ represents the assignment of the event to one of two classes.
\begin{Exa}
    In \TFC\ the input is an instant in a driving situation, and this input should be assigned to either the class alert, or the class not-alert. The input is represented by an 30-dimensional vector and the assignment to a class is represented by $t=0$ meaning not-alert and $t=1$ alert.
\end{Exa}
To make the prediction possible, a trainingset $\mathcal{T}$ is given consisting of $n$ pairs of input vectors and corresponding, known outcomes.
\[
    \mathcal{T} = \bigl\{\:(\ve{x}_i, t_i)\:\bigr\}
\]
Based on this trainingset, a classification rule $f$ is learned, that can predict the outcome, of yet unknown input vectors. Therefore a classification rule is a function $f\,:\,\R^d\to\{0,1\}$, that is used to make the prediction $t=f(x)$ on a new input. \par
A common way to obtain a classification rule is by the Bayes classification rule. First define $p_k(\ve{x})$ as
    \[
        p_k(\ve{x}) = P(t=k|\ve{x}),\quad k\in\{0,1\}
    \]
    and then the Bayes classification rule is defined by
    \begin{definition}\label{def:bayes-rule}
        Let $\alpha\in[0,\infty[$. The Bayes classification rule $f^*$ is given by
        \[
            f^*(\ve{x}) = \begin{cases}
                1 & \text{if}\quad \alpha p_1(\ve{x})>p_0(\ve{x}) \\
                0 & \text{else}
            \end{cases}
        \]
    \end{definition}
    Since $p_0(\ve{x})=1-p_1(\ve{x})$, the Bayes classification rule can be written
    \[
        f^*(\ve{x}) = \begin{cases}
            1 & \text{if}\quad p_1(\ve{x}) > \frac{1}{1+\alpha} \\
            0 & \text{else}
        \end{cases}
    \]
    By the Bayes classification rule we have got a theoretical classification rule. In practice the posterior class probabilities $p_0(\ve{x})$ and $p_1(\ve{x})$ are unknown, so the trainingset must be used to find some approximation for these probabilities. The problem of approximating a probability distribution from a given data set is a classic statistical problem. The next section gives one way to solve this problem.

\section{Parametric models and maximum likelihood}\label{sec:parametric-models-and-likelihood}
Two distinct ways to approximate the probability distribution $\Ptx$ is using either a parametric or a non-paramtric approach. The focus in this report is on a paramtric approach. The distribution $\Ptx$, is therefore approximated by a distribution $\PtxHat$ restricted to a class of distributions
\[
    \mathcal{F} = \Bigl\{\,f_{\ve{w}}(t|\ve{x})\,\Bigr\}
\]
where the distributions $f_{\ve{w}}\in\mathcal{F}$ are uniquely identified by their parameter $\ve{w}=[w_1, w_2, \dots, w_k]^T$. By restricting the approximating distribution $\PtxHat$ to the set $\mathcal{F}$, the problem of approximating $\Ptx$ reduces to the problem of estimating the parameter $\ve{w}$.

\subsection{Maximum likelihood}
There exist many different techniques to estimate the parameter $\ve{w}$ in a parametric model\footnote{See eg. \citet[Sec.9]{wasserman04}}




\section{Logistic Regression}




\section{The binary classification problem}\label{sec:binary-classification-problem}
In a binary classification problem, a set of $n$ inputs are given along with a set of $n$ corresponding class labels. These two sets are called the trainingset. Based on this trainingset, a procedure is \mydef{learned}. The procedure should predict the class of a new input, with as few errors as possible. The input is represented by a $d$-dimensional vector 
\[
    \ve{x} = \begin{bmatrix}
        x_1 \\ 
        x_2 \\ 
        \vdots \\
        x_d 
    \end{bmatrix}
\]
and the predicted class is represented by a binary variable $t\in\{0,1\}$. The trainingset are then given by a set 
\[
    \mathcal{T} = \bigl\{\:(\ve{x}_i, t_i)\:\bigr\}_{i=1}^n 
\]
In the Ford Challenge eg., the input is a driving instant and the classes are alert and not alert. The driving instant is represented by a 30-dimensional vector of physiological, environmental and vehicular features, and the alert/not-alert classes are represented by respectively $t=1$ and $t=0$. \par
A simple way to mathematically express the binary classification problem is: Given a trainingset $\mathcal{T}$, a discriminant function $f : \myreal^d \to \{0,1\}$ must be learned such that errors on future inputs are minimized. Although this is a simple description, and concrete classifier methods using this approach exists\footnote{Eg. \citet[p.181]{bishop}}, it is preferred \citep[p.43]{bishop} to use a probabilistic description. In this perspective the classification consists of two separate steps, namely an inference step and a decision step. \par
    In the inference step, the problem of determining the probabilities $P(t|\ve{x})$\footnote{In another approach the $P(t|\ve{x})$ isn't determined directly. Instead $p(\ve{x}|t)$ and $P(t)$ are determined and then using Bayes Formula $P(t|\ve{x})$ is calculated. This is called generative modelling.}, are solved\mytodo{Should i use $P$ for mass distribution and $p$ for density distribution?}. These probabilities are called the posterior class probabilities. \par
    In the decision step a loss table is set up describing the relative costs of predicting a class of an input given its true class. For The Ford Challenge, the loss table could eg. be as shown in table~\ref{tbl:loss-table}.
    \begin{table}
        \centering
        {\small\sffamily
        \begin{tabularx}{6cm}{ l | X X }
            {\footnotesize True $\backslash$ Predict} & Not Alert & Alert \\\hline
            Not Alert & 0 & 100 \\
            Alert & 1 & 0
        \end{tabularx}
        }
        \caption{An example of a loss table for The Ford Challenge classification problem.}
        \label{tbl:loss-table}
    \end{table}
    The information in the loss table can be represented as a loss matrix $L$. In our example this means that eg. $L_{01}=100$. Using the posterior class probabilities $P(t|\ve{x})$ a decision about the predicted class label of an new input $\ve{x}_0$ can be made by the rule
    \[
        \min_j \sum_{k} L_{jk}P(t=k|\ve{x}_0)
    \]
    where $j,k\in\{0,1\}$. In words the decision step chooses the class label that minimizes the expected loss\mytodo{Is this the right interpretation?}. \par
    What is gained from making the two steps explicit? If we only have a discriminant function $f$, and the loss table is changed, we need to train a whole new discriminant function, using the trainingset again. If we instead have determined the posterior class probabilities, we only need to update the loss matrix, and don't have to retrain anything. Another reason is a common sense argument. When a class label is predicted for a new input, there will always be some level of uncertainty in the prediction. In the Ford Challenge the class label is predicted for a 30-dimensional input vector, but this vector is only a representation of the real input; namely an instant in a driving session. There is no doubt that some details that could influence the alertness of the driver aren't included in the 30 measurements, and this gives an uncertainty in the prediction. Probability is what is used to mathematically express uncertainty, and it therefore seems natural to take a probabilistic view on the classification problem. \par
    The theory presented in this section has been on a general level. Two questions of practical importance needs to answered though. First of all, how can the posterior class probabilities $P(t|\ve{x})$ be determined from the training data? And secondly, how can the performance of the classifier on future inputs be determined, when training the classifier? The answer to the first question is delayed till the sections about specific classifier methods, and the second question is answered in the next section.
    \begin{definition}
        Anders
    \end{definition}


\section{Measuring classifier performance}\label{sec:classifier-performance}




\section{Logistic Regression}\label{sec:logistic-regression}




\section{Feed Forward Neural Network}\label{sec:feed-forward-neural-network}




\section{AUC}\label{sec:theory:auc}
Mentioning critiques of AUC?


\chapter{Recreating winning approach}\label{sec:recreating}
Here I describe how I have tried to recreate the winning approach. How I measure performance. The problem that I do not have access to the test data set used by Inference. My results. The scikits.learn library that I have used.

\chapter{Improving the winning approach}
Here I try to improve the winning approach, by doing my own feature selection. Forward selection. Lasso. Cross validating with Lasso. Using window instead of running.

\chapter{Other classification methods}
Here I describe some alternatives to the logistic regression used by Inference. Hoping to get a result or two from SVM or Neural Network.

\part{Workflow and discussion}
\chapter{Workflow and tools}\label{chp:tools}
Describing software used, workflow using github, writing sessions, evaluate my performance, mention the small improvement of roccurve method.

%\include{chapters/describing-the-workflow}
\chapter{Discussion}

In this chapter some of the discussions from throughout the report will be resumed. \par

Starting with the data exploration, one of the early surprises was the (lack of) data quality. This was hinted at already in the calculation of summary statistics. Some features had datapoints many, many standard deviations from the mean, and some features were simply zero throughout the dataset. Although summary statistics isn't that informative it was a simple way to get started with the data exploration. \par

Calculating the number of unique values within a trial for each feature, was an effective way to get an idea of the datatype of the features. Also the unique values made it clear, that many features had trials where they were constant. \par

Plotting the different features for various trials gave yet another confirmation that this was indeed a real-world dataset. When the time came for outlier detection it was found that it was difficult to make a rational criteria for what data to exclude. It seems to have been the right choice not to remove any data, as later models turned out to depend on two well behaved features \fn{E9} and \fn{V11} as well as two aggregated features \fn{sdE1}, \fn{sdE5}. Had some trials been removed, useful data for these features would have been discarded. \par

Creating scatterplots of all pairs of features didn't really reveal anything, except for the inverse relationship between two pairs of features. The Principal Components Analysis didn't reveal much either except for a possible cluster of not-alert points that were detected. This information could maybe have been used to get better classification. \par

In the data modelling part of the project, the winning model was quickly recreated. It is interesting to note that a competition with 180 participants, is won by using a simple logistic regression on only 3 out of 30 features. Had the features been some ingenious combination of other features, it wouldn't be surprising, but two of the features of the winning model turned up in the simple forward selection done in section~\ref{sec:forward-selection}. Regarding the forward selection, the AUC score was used as the measure for deciding which features to include. This is not the standard way \citep{meetings-morten} of doing a forward selection, but it seems to be a logical choice if the classifier performance in the end is measured by the AUC score. Some confirmation of this argument is seen by the fact, that the forward selection revealed a feature \fn{sdE1} that consistently gave higher AUC score than the feature \fn{sdE5} used in the winning model. The conclusion is weakened a bit by the fact, that the performance of the forward selection model on the real Ford testset is unknown. Perhaps the winner also first chose \fn{sdE1} but then found that \fn{sdE5} performed better on the Ford testset. \par

The attempts to further improve the performance, by training a neural network wasn't that succesful. Most of the results were similar to the results achieved by the logistic regression. Only the run with the forward selected features, 3 hidden layers, and 20 iterations, showed signs of improvements. Had the network been trained for maybe 200 iterations\footnote{An undocumented run with 50 iterations didn't give any improvements though}, further improvements might have been detected. As a final note about the neural network modelling, it was interesting to watch the poorer performance of the network with 5 hidden units, trained for 20 iteration, and using the forward selected features. The performance can be explained by overfitting of the network to the training data, but further experiments should be done before concluding anything. \par

Part of the problem statement was to set up a working environment for data analysis. This was done, and some structure was put on the work process, that helped controlling the chaos that seems inevitable. Also some great libraries for scientific computations and machine learning was found and was put to good use throughout the project.

\chapter{Conclusion}
More bla, bla, bla.

%\appendix

%%%%%%%APPENDIX CHAPTERS INCLUDE%%%%%%%%%%%%%%%%%%%%%%%%%%%%%%%%%%%%%%%%%%%%%%

\newcounter{alphasect}
\addtocounter{alphasect}{1}
\renewcommand{\thechapter}{\Alph{alphasect}}
\chapter{Appendices}
A little introduction and then a new page
\vspace{\stretch{1}}
\pagebreak

\section{Source code for calculating common statistics}\label{app:source-common-statistics}
\lstinputlisting{../sessions/9-data-exploration/src/calculate_summary_statistics.py}
\vspace{\stretch{1}}

\pagebreak

\section{Results of calculating common statistics}\label{app:result-common-statistics}
{\small\sffamily
\begin{python}
    import scripts.commonstats_table as c; c.render('../sessions/9-data-exploration/src/summary_statistics.json')
\end{python}
}


% Appendix A, B, ...
%\include{Appendix1}
%\include{paper1}
%\include{paper2}


\backmatter

\chaptermark{Bibliography}
\renewcommand{\sectionmark}[1]{\markright{#1}}
\sectionmark{Bibliography}

%%%%%%%BIBLIOGRAPHY INCLUDE%%%%%%%%%%%%%%%%%%%%%%%%%%%%%%%%%%%%%%%%%%%%%%

\bibliography{bibdb}    % Bibliography
%\bibliographystyle{plainnat}
\bibliographystyle{anders}


\end{document}
