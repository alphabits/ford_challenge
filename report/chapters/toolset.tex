\chapter{Workflow, tools and project evaluation}\label{chp:tools}
In this chapter the workflow used while working on this project, is introduced. As the workflow is tightly connected with the tools (software) used, these will be introduced when they are first mentioned. Before the start of this project, I\footnote{The narrative style changes in this chapter, as some personal thoughts and opinions will be expressed} decided on some ``metagoals'' for this project
\begin{itemize}
    \item I should spend some time on setting up a workflow that can be used for future data analysis projects.
    \item I should be able to set up my working environment on any linux/mac within an hour.
    \item All project related files should be managed by a version control system
    \item Anyone that wishes to work on my project should be able to start working on it, within an hour or two.
    \item My workflow should encourage me to write documentation of all the work done
    \item Find machine learning software libraries that could be useful for many years into the future
\end{itemize}
Reflecting on well the different goals was achieved is the subject of the rest of this chapter.


\section{Workflow}
In this section the workflow that was used throughout this project will be described. A simple session concept, that turned out to work great, is introduced along with some suggestions for improvements.


Git and GitHub
The Session Concept
Latex packages
Python/NumPy/SciPy
Sphinx/Documentation
scikits.learn
pybrain
Automation/Fabric

Things to look more into:
C programming
R


Should be able to work on bith my mac and linux
Anyone that wishes to work with my project, should be able to get up and running within an hour.



This chapter exists in a gray zone between the main report and the appendices. It isn't part of the main report, as the style changes, as more personal beliefs are expressed and reflection about the project are m


Describing software used, workflow using github, writing sessions, evaluate my performance, mention the small improvement of roccurve method.
