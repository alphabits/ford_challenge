\chapter{The Ford Challenge}
In this chapter, the ford competition that is the basis for this report, is introduced. Before introducing the ford challenge, a short overview of other online machine learning competitions is given. As part of introducing the ford competition, the kaggle.com website that hosted the ford competition is presented, and the data set used in the ford competition is described in detail. But as metioned the chapter starts with a short overview of other online machine learning competitions.

\section{Other online machine learning competitions}
To get a little perspective on the ford challenge, this section gives a short review of some past and present online machine learning competitions.\par
One of the most talked about competitions, may very well be the Netflix Prize. The Netflix competition was launched on October 2, 2006 and the aim of the competition was to predict how users would grade new movies, based on a large dataset of previous grades. Why did the Netflix Prize gather a lot of attention? First of all the grand prize was \$1M, which is a lot of money. And secondly the dataset was huge, consisting of 100,480,507 ratings given by 480,189 users, leaving room for a lot of interesting new techniques to be tested. \par
When the Netflix Prize was awarded on September 18, 2009, 5169 different teams had submitted at least one entry to the competition. The winning team consisted of three different teams, that at one point decided to team-up and compete as a join-team. \par
Following the 


\section{The ford competition}


\subsection{The data set}


