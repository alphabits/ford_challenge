\documentclass[12pt]{article}
\linespread{1}

\usepackage{geometry} % see geometry.pdf on how to lay out the page. There's lots.
\usepackage[utf8]{inputenc}
\usepackage{array}
\usepackage{amsmath}
\usepackage{amsfonts}
\usepackage{amssymb}
\usepackage{graphicx,float}
\usepackage{pstricks}
\usepackage{pst-node}
\usepackage{pst-tree}
\usepackage{wrapfig}
\usepackage{caption}
\usepackage{multirow}
\usepackage{fouriernc}
\usepackage[scaled]{beramono}
\usepackage{listings}
\lstset {                 % A rudimentary config that shows off some features.
    language=Python,
    basicstyle=\footnotesize\ttfamily, % Without beramono, we'd get cmtt, the teletype font.
    commentstyle=\textit, % cmtt doesn't do italics. It might do slanted text though.
    keywordstyle=\bfseries,
    framextopmargin=10pt,
    framexbottommargin=10pt,
    framexleftmargin=10pt,
    framexrightmargin=10pt,
    xleftmargin=10pt,
    xrightmargin=10pt,
    backgroundcolor=\color{mygray},
    frame=single,
    tabsize=4            % Or whatever you use in your editor, I suppose.
}
% \usepackage[charter]{mathdesign}
% \usepackage{lmodern}
\usepackage[normalem]{ulem}
\geometry{a4paper} % or letter or a5paper or ... etc
% \geometry{landscape} % rotated page geometry

\usepackage{url}
\usepackage{natbib}
\renewcommand\bibsection{}
\bibliographystyle{plain}

\makeatletter
\renewcommand*\env@matrix[1][*\c@MaxMatrixCols c]{%
  \hskip -\arraycolsep
  \let\@ifnextchar\new@ifnextchar
  \array{#1}}
\makeatother

% See the ``Article customise'' template for come common customisations

\newtheorem{mydef}{Definition}
\newtheorem{saet}{Sætning}
\newtheorem{eks}{Eksempel}

\newcommand\myimp{\quad\Leftrightarrow\quad}
\newcommand\half{\frac{1}{2}}
\newcommand\myvec[1]{\mathbf{#1}}
\newcommand\mymod[1]{\ (\text{mod }#1)}
\newcommand\myreal{\mathbb{R}}
\newcommand\mynatural{\mathbb{N}}
\newcommand\myint{\text{int}}
\newcommand\norm[1]{||\,#1\,||}
\newcommand\arr[1]{\langle#1\rangle}
\newcommand\vrb[1]{\texttt{\small#1}}
\newcommand\floor[1]{\lfloor#1\rfloor}
\newcommand\ceil[1]{\lceil#1\rceil}
\newcommand\mytr[1]{\small\Tr{#1}}

\usepackage{fancyvrb}
\DefineShortVerb{\@}

\usepackage{tabulary}
\newcolumntype{y}{>{\centering\arraybackslash}R}

\usepackage{color}
\definecolor{mygray}{RGB}{230,230,230}

\usepackage{fancyhdr}
\pagestyle{fancy}
\lhead{\small\textit{Hjemmeopgave 2 - 02105 Algoritmer og datastrukturer - Anders Hørsted (s082382)}}
\rhead{\thepage}
\chead{}
\lfoot{}\cfoot{}\rfoot{}

\title{Hjemmeopgave 2}
\author{Anders Hørsted (s082382) -- Hold 1 -- 10-03-2011\\\small 02105 Algoritmer og datastrukturer}
%\author{}
\date{} % delete this line to display the current date

%%% BEGIN DOCUMENT
\begin{document}
    \begin{figure}[htbp]
        \centering
            \psset{treenodesize=.2cm,levelsep=25pt} 
            \pstree{\mytr{$n$}}{% 
                \pstree{\mytr{$n-1$}}{% 
                    \pstree[linestyle=dashed]{\mytr{$n-2$}}{% 
                        \pstree[linestyle=dashed]{\mytr{$n-k$}}{% 
                            \mytr{$1$}
                            \Tn
                        }
                        \mytr{ }
                    }
                    \pstree[linestyle=dashed]{\mytr{$n-3$}}{% 
                        \mytr{ }
                        \mytr{ }
                    }
                }
                \pstree{\mytr{$n-2$}}{% 
                    \pstree[linestyle=dashed]{\mytr{$n-3$}}{% 
                        \mytr{ }
                        \mytr{ }
                    }
                    \pstree[linestyle=dashed]{\mytr{$n-4$}}{% 
                        \mytr{ }
                        \mytr{$n-2k$}
                    }
                }
            }
        \caption{Træ over operationer udført ved rekursiv beregning af $F_n$}
        \label{fig:fibotree}
    \end{figure}
\end{document}
